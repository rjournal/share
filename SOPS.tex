\documentclass[11pt]{article}
\usepackage{url}
\begin{document}

\title{Standard Operating Procedures for The R Journal}
\author{Martyn Plummer}
\date{6 December 2012}
\maketitle

\section{Editorial Board}

The board consists of four editors. Each editor serves a term of four
years, starting on 1 January. The editorial board handles submissions
of Contributed Research Articles.

In addition to the board members, there are three topic editors who
handle submissions under the sections ``Programmer's Niche'', ``Help
Desk'' and ``Book Reviews''. Topic editors serve as long as they wish.

\subsection{Editor in Chief}

The editor in the third year of their term acts as Editor-in-Chief
(EIC). The EIC takes primary responsibility for
\begin{itemize}
\item Production of two journal issues per year, in January and December.
\item Managing the flow of submissions, ensuring that these are
  handled in a timely manner.
\item Maintaining the R Journal web site \url{journal.r-project.org}
\item Maintaining the LaTeX style files and tools used to manage the
  journal.
\end{itemize}

\subsection{Advisory Board}

An advisory board consisting of former editors of the Journal may be
consulted for advice on strategic issues.  The email address of the
advisory board is \url{r-journal-advisory@r-project.org}. The R
Journal Advisory list is archived.

\subsection{email}

All editors are members of the mailing list
\url{r-journal-editors@r-project.org}, which is a private list used to
discuss current editorial issues. The list is not archived.
 
The board members have an R-project email alias
\url{firstname.lastname@r-project.org} which may be used for R Journal
correspondence.  However, access to the \verb+r-journal-editors+ mailing list
is not possible using the \verb+r-project.org+ alias.

The board members are also members of the \verb+r-journal-advisory+
mailing list.

\subsection{subversion}

Board members have access to two subversion (SVN) repositories
\begin{description}
\item[Rjournal-adm] The \texttt{Rjournal-adm} repository used for
  submission management. The repository is located at
  \url{https://svn.r-project.org/Rjournal-adm/trunk}.
\item[Rjournal] The \texttt{RJournal} repository for the R Journal web
  page. The repository is located at
  \url{https://svn.r-project.org/Rjournal/trunk}) Changes to the
  \texttt{Rjournal} repository do not immediately affect the web site
  \url{journal.r-project.org} but are copied via a cron job twice per
  day.
\end{description}
Topic editors do not have subversion access.

\subsubsection{LaTeX environment}

Editors should set up the \texttt{TEXINPUTS} environment variable to
point to the subdirectory \texttt{share} in the working copy of the
\texttt{Rjournal-adm} repository, e.g., by adding the following lines to
\verb+.bash_profile+

\begin{verbatim}
TEXINPUTS=/path/to/Rjournal-adm/share/:$TEXINPUTS
export TEXINPUT
\end{verbatim}

This ensures that the current version of \texttt{Rjournal.sty} is used
along with any configuration files.

\section{Submissions}

All submissions are coordinated through entries in the \texttt{index.dcf}
file. Each entry has the following fields:
\begin{description}
\item[Refno] Unique reference number in the form YYYY-n for the nth
  submission in year YYYY.
\item[Authors] List of full author names
\item[Email] Email address of corresponding author
\item[Title] Title of submission
\item[Editor] Handling editor
\item[Reviewer1] Name of first reviewer
\item[R1email] Email address of first reviewer
\item[Reviewer2] Name of second reviewer
\item[R2email] Email address of second reviewer
\item[Status] Current status, taking one of the following values:
  needs editor; needs reviewers; under review; accepted; rejected;
  and, for published articles, the corresponding issue number
\item[Comments] Free-form field used to make notes on the submission and
  its history.
\end{description}

Since \texttt{index.dcf} is shared, great care must be taken not to
overwrite changes made by other editors. For example, after modifying
\texttt{index.dcf}, go through the following steps:

\begin{verbatim}
svn update           #Update to latest version
svn status index.dcf #Check for conflicts
svn diff index.dcf   #Confirm changes
svn commit index.dcf #Commit changes
\end{verbatim}

\subsection{New submissions}

Submissions are normally sent to the EIC. However, any member of the
board may handle a new submission if they follow the same procedure.

\begin{itemize}
\item Create an entry in \texttt{index.dcf}.
  \begin{itemize}
  \item Assign the next available submission number 
  \item Fill the  fields authors, email, and title from the relevant
    information in the submission.
  \item Set the status field to ``needs editor''
  \end{itemize}
\item Acknowledge receipt of the submission to the authors, giving
  reference number.  Note dates of receipt and acknowledgement in the
  \texttt{Comments} field of \texttt{index.dcf}.
\item Create a subdirectory for the submission under the Submissions
  directory.  The subdirectory should contain 
  \begin{description}
  \item[RJwrapper.pdf]: The PDF file to be sent to reviewers
  \item[The author file]: The LaTeX file included by \texttt{RJwrapper.tex}
    containing the LaTeX source for the article
  \item[Graphics files]: Any files included as images in the article.
  \item[The bibliography file]: Authors must use BibTeX to create the
    bibliography. This ensures that the bibliography conforms to the 
    journal style and greatly facilitates corrections during copy-editing
    and proof-reading.
  \end{description}
\item The article subdirectory should {\bf not} contain the following files:
  \begin{description}
  \item[RJournal.sty] The editors should use only the current style
    file in \texttt{share/RJournal.sty}, not the version submitted by
    the author, which may be out of date.
  \item[Zip files] Zip files are binary files and are not amenable to
    version control using SVN. Only the contents of zip files should
    be checked in.
  \item[Built files] Any intermediate files created by pdflatex while
    building \texttt{RJwrapper.pdf}.
  \end{description}
\end{itemize}

\section{Handling Editor}

The EIC regularly (e.g., every month) makes a summary of submissions
and invites the members of the board to take on submissions as
handling editor. This editorial duty is shared equally among members
of the board, including the EIC.

The handling editor fills in their name in the ``editor'' field and
changes the status of the submission to ``needs reviewers'' The
handling editor is responsible for keeping the \texttt{index.dcf} file and the
files in the repository up to date for the given submission.

\section{Peer review}

The handling editor selects and invites reviewers. Each submission
should have at least two reviewers. Exceptionally, a third reviewer
may be invited, in which case additional fields are created in the
\texttt{index.dcf} file. Under normal circumstances, the reviewers
remain anonymous to the authors.

The reviewers should be given a clear deadline for the return of
reviews.  This should an absolute, not a relative deadline (i.e., a
date, not ``in four weeks''). Suitable review times are
\begin{itemize}
\item Four weeks for most articles
\item Two weeks for short articles (under 5 pages) or those containing
few formulae and little R code
\item Six weeks for longer or more technical articles
\end{itemize}
The dates on which reviews are expected should be noted in the
\texttt{Comments} field so that the EIC can see the current status.

When reviewers have agreed, their names and email addresses should
be entered into the \verb+Reveiwer1/2+ and \verb+REmail1/2+ entries.
The \texttt{status} field should be changed to ``under review''.

The handling editor is responsible for communication with the authors
and reviewers. Email correspondence with the reviewers and the authors
is not copied to the \verb+r-journal-editors+ mailing list (which is
reserved for discussions between editors). However, copies of the
correspondence may be checked into the submission directory under the
subdirectory \texttt{correspondence}.

Copies of the reviews and, when applicable, the authors' response to
the reviewers should be checked into the submission subdirectory with
appropriate file names (e.g. \texttt{R1.pdf}, \texttt{R2.txt},
\texttt{Response.pdf})

\section{Resubmissions}

Only the current versions of the submission files are of interest.
When a revised version of the paper is received, the files for it {\bf
  replace} those of the original version. Editors {\bf MUST NOT}
create multiple copies of submission files in the submission directory
as this creates an ambiguous situation for the EIC when the time comes
to publish the issue.

Note that any file in the working copy can be temporarily reverted to
a previous version with \verb+'svn update -rXXX'+ where \verb+XXX+ is
the number of the repository version containing a previous revision.
Such revision numbers can be found with
\verb+'svn log <filename>'+. The next call to \verb+'svn update'+ will
restore the working copy to the current version.

\section{Final disposition}

The handling editor is responsible for deciding whether the paper is
accepted or rejected, and whether a revision is requested.

When an article is rejected, or if the article is withdrawn by the
authors, the handling editor performs the following actions:
\begin{itemize}
\item The article subdirectory is moved from \texttt{Submissions} to
  \texttt{Rejected} using \verb+svn move+
\item The article status is updated to ``rejected'' in the \texttt{index.dcf}
  file.
\end{itemize}
When an article is accepted for publication:
\begin{itemize}
\item The article subdirectory is moved from \texttt{Submissions} to
  \texttt{Accepted} using \verb+'svn move'+
\item The article status is updated to ``accepted'' in the \texttt{index.dcf}
  file.
\item The handling editor sends a notification to the \verb+r-journal-editors+
  mailing list to inform the editors that the article is accepted.
\end{itemize}

\section{Copy-editing and online preview}

Accepted articles are eligible for online preview on the R Journal web
site.  Accepted articles must first go through copy-editing process by
a member of the board (preferably not the handling editor). Copy
editing is shared equally among members of the board, including the
EIC.

Copy editing ensures that 
\begin{itemize}
\item The article conforms to the journal style (see the file
  \verb+proofreading_checklist.txt+)
\item English spelling and grammar are correct.
\item There are no layout issues (e.g. non-floating figures and tables
  leaving large amounts of white space
\item The bibliography is correct.
\end{itemize}
When copy editing is complete, the responsible editor sends a message
to the \verb+r-journal-editors+ mailing list.

\subsection{Creating an online preview}

The EIC creates an online preview for accepted articles, working
in the \texttt{Rjournal} repository. Creating an online preview
requires the Perl template toolkit (See ``Publishing a New issue'')

\begin{enumerate}
\item Create a new subdirectory of \verb+html/accepted+ using the
  reference number of the article (\verb+refno+)
\item Copy \texttt{RJwrapper.pdf} from the \texttt{Rjournal-adm}
  repository to the \verb+html/accepted/refno/+ directory and rename it
  after the authors (e.g., \verb|Eugster+Schlesinger.pdf|)
\item Create an entry in \texttt{accepted.txt} containing 
  \begin{itemize}
  \item The article title
  \item The author names
  \item The path to the PDF file (e.g.,
    \verb|accepted/2012-06/Eugster+Schlesinger.pdf|)
  \end{itemize}
  separated by ``|'' characters.
\item Remove the file \verb+html/accepted.html+ and run
  \verb+'ttree -f RJ.ttreerc'+
\item Ensure that the new version is online by opening the file 
\verb+html/index.html+ in a web browser and navigating to the
``Accepted Articles'' page.
\item Add the new files to the SVN repository and check in the updates.
\end{enumerate}

\section{Preparing a new issue}

The EIC should begin the preparation of a new issue at least 3 weeks
before the due publication date.

The EIC creates a new sub-directory of Proofs using the naming
convention \verb+<year>-<issue>+ (e.g., 2012-2). The following files
should be copied into this directory

\begin{itemize}
\item The following files can be copied from the previous issue directory
\begin{itemize}
\item ch.tex
\item RDsub.tex
\item Rlogo.png
\item Rlogo-4.png
\end{itemize}
\item The latest \texttt{RJournal.sty}
\item A copy of the LaTeX wrapper file
  \verb+share/RJournal_template.tex+ renamed as
  \verb+RJournal_<year>-<issue>.tex+. In the sequel, this is called
  the {\em issue file}.
\end{itemize}

The following lines of the issue file should be edited to match the
current issue.
\begin{verbatim}
\volume{v} %Replace v with current volume number
\volnumber{n} %Replace n with current issue number
\date{Month Year} %Replace with current month (in letters) and year (yyyy)
\end{verbatim}

If the editorial board has changed since the last issue, then the
\verb+RJournal_template.tex+ file should be updated.

\subsection{Editorial}

The editorial \texttt{editorial.tex} is written by the EIC and is
usually one page long.  There should be page breaks after the
editorial that the first Contributed research article begins on an
odd-numbered page.

\subsection{Contributed Research Articles}

The articles to be published are selected by the EIC from among the
currently accepted articles. Articles with a lower submission number
are usually published first, leaving more recently submitted articles
in the queue, even if they were accepted earlier.

\subsubsection{Moving the source files}

Articles to be published are moved from the Accepted directory, e.g.
\begin{verbatim}
svn move Accepted 2011-05 Proofs/2012-2
\end{verbatim}

\subsubsection{Including an article in the issue file}

Each article is included in the issue file as follows:
\begin{verbatim}
\begin{article}
\setkeys{Gin}{width=\columnwidth}
\subimport{<refno>/}{<Authorfile.tex>}
\end{article}
\newpage
\end{verbatim}
where \verb+<refno>+ is the reference number of the article
(e.g. \texttt{2011-05}) and \verb+<Authorfile.tex>+ is the {\em author
  file} -- the file containing the LaTeX source that is usually named
after the authors (e.g., \texttt{HaNohLee.tex}. The \verb+\subimport+
command comes from the import package and allows source files to be
included from a sub-directory.  Note that the trailing slash after
\verb+<refno>+ is required: if omitted it will lead to a ``file not
found´´ error message.

The line 
\begin{verbatim}
\setkeys{Gin}{width=\columnwidth}
\end{verbatim}
sets the figure width to be the same as the column width. However, if
the author sets the figure width explicitly by using the
\texttt{width} option to \verb+\includegraphics+ then this line will
cause the figures to be sized incorrectly and it should be omitted.

If the article requires special LaTeX macro definitions or uses
special packages, these are found in the preamble of the
corresponding \texttt{RJwrapper.tex}. These lines should be copied
from \texttt{RJwrapper.tex} into the issue file below the line:
\begin{verbatim}
%% packages, etc., for particular articles 
\end{verbatim}

\subsubsection{Bibliography}

Each article has its own self-contained bibliography.  If the authors
use BibTeX, then the bibliography should be folded into the author file
as follows:
\begin{verbatim}
pdflatex RJwrapper
bibtex RJwrapper
pdflatex RJwrapper #Repeat as necessary
\end{verbatim}
Then uncomment the line \verb+\bibliography{<bibfile>}+ in the author
file and paste in the contents of \texttt{RJwrapper.bbl} into the
issue file.

It may occur that two articles use the same key to identify a
bibliography entry. This will result in a warning message ``Package
natbib Warning: There were multiply defined citations´´.  The relevant
citations can be found by inspecting the output from
\texttt{pdflatex}.  This issue should be resolved by renaming the key
in one of the author files.

\subsection{Themed articles}

Themed articles appear under the following sections of the Journal:
\begin{itemize}
\item From the core
\item Programmer's Niche
\item Helpdesk
\item Book Reviews
\end{itemize}
They are included in the issue file in the same way as contributed
research articles. However, if there are no themed articles in a given
section, then the corresponding lines in the issue file starting with
\verb+sectionhead+ and \verb+\addtocontents{toc}+ should be removed or
commented out.

\subsection{News and Notes}

The source files for News and Notes are
\begin{itemize}
\item \texttt{cran.tex} for Changes on CRAN. This file is provided by
  Kurt Hornik and Achim Zeileis. The EIC writes to them a few weeks
  before publication.
\item \texttt{foundation.tex} R Foundation News, also provided by Kurt Hornik
\item \texttt{biocch.tex} News from the Bioconductor project. This is
  provided by Martin Morgan. The EIC also writes to him a few weeks
  before publication.
\item \texttt{NEWS.tex} Changes in R.  This is created from the current
  \texttt{NEWS.Rd} file as described below.
\end{itemize}

\subsubsection{Creating NEWS.tex}

\begin{enumerate}
\item Copy \texttt{NEWS.Rd} from the latest released version of R
\item Remove any sections that were already included in the previous issue.
\item Change the first line (which defines \verb+\PR+ to do something
complicated) to just \verb+\newcommand{\PR}{PR##1}+
\item Run \verb+R CMD Rdconv -t latex NEWS.Rd > NEWS.tex+
\item If any undefined \verb+Rd+ macros turn up, update Rdsub.tex (most likely
by stealing code from \verb+share/texmf/tex/latex/Rd.sty+ in the R sources
\item Ideally, fix typos, missing markup, etc. in the {\em original}
\texttt{NEWS.Rd} and redo steps 4-5.
\item Any remaining formatting (Overfull boxes, etc.) has to be
  handled {\em ad hoc}
\end{enumerate}
Note that \texttt{NEWS.Rd} is typically UTF-8 encoded.  Any non-ASCII
characters have to be converted to ASCII using the appropriate LaTeX
macros, but this must be done in \texttt{NEWS.tex} and not in the
original \texttt{NEWS.Rd} file.

\section{Pre-publication}

\subsection{EIC tasks}

Before launching the pre-publication process, the EIC should verify
the following:
\begin{itemize}
\item A PDF version of the issue file can be built without errors or
  warnings.
\item All LaTeX source files are checked in to the SVN repository and
  are up to date so that the associated editors can rebuild the issue
  if necessary.
\item The status of each articles included in this issue is updated in
  the file \texttt{index.dcf} to the current issue number (e.g.,
  \verb+Status: 2012-2+).
\end{itemize}

Pre-publication proceeds as follows:

\begin{enumerate}
\item The EIC checks in a copy of the PDF built from the issue file
  into the ``RJournal´´ SVN repository in the sub-directory
  \texttt{prepub}. From there it is publicly accessible via
  \url{http://svn.r-project.org/RJournal/trunk/prepub/RJournal_year-issue.pdf}.
\item The EIC informs the other editors via a message to
  \url{r-journal-editors@r-project.org} that the pre-publication
  version is ready giving the URL.
\item The EIC distributes proof-reading duties between the associate
  editors. An editor should not normally proof read an article that
  they previously handled as an editor
\item The EIC sets a publication date, and a clear deadline for the
  return of corrected proofs.
\end{enumerate}

\subsection{Proof Reading}

The associate editors check the articles assigned to them for errors
and make necessary corrections. They then write to the authors

\begin{itemize}
\item The AE gives the URL for the pre-publication version of the
  current issue to the authors.
\item The AE attaches the current latex source (current author file
  and \texttt{RJwrapper.tex} file) so that the authors can make any
  necessary changes to the latest version.
\item The AE instructs the authors to comment only on their own article.
\item The AE informs the authors of the deadline for the return of
  corrected proofs.
\end{itemize}

\section{Publishing a new issue}

To publish a new issue, the EIC works with the ``RJournal'' SVN
repository. The EIC must have the Perl template toolkit
(\texttt{ttree}) installed.  See
\url{http://template-toolkit.org/download/index.html}. On Fedora, this
is provided via the RPM package ``perl-Template-Toolkit''.

\subsection{Removing online versions}

All articles included in the current issue should have their 
pre-publication online versions removed from the web site. 
\begin{enumerate}
\item The appropriate sub-directories of \verb+html/accepted+ are
  deleted using \verb+'svn remove'+.
\item The corresponding entries in the file \texttt{accepted.txt} are 
  removed.
\item The listing of accepted articles is refreshed by deleting the
  file \texttt{html/accepted.html} and running the command
  \verb+'ttree -f RJ.ttreerc'+
\end{enumerate}

\subsection{Creating the issue}

The process of creating a new online issue is simplified by the
R script \texttt{issue2web.R} which is invoked from the trunk
directory of RJournal as follows:
\begin{verbatim}
 R CMD BATCH '--args <year-issue> </path/to/Rjournal-adm> publish' issue2web.R
\end{verbatim}
where \verb+<year-issue>+ is the current issue number
(e.g. \texttt{2012-2}) and \verb+</path/to/Rjournal-adm>+ is the
directory containing the trunk of the Rjournal-adm SVN repository. 

This creates two new directories
\begin{verbatim}
/templates/archive/<year-issue>
/html/archive/<year-issue>
\end{verbatim}
The keyword ``publish'' ensures that these new directories are
added to the SVN repository.

The EIC reviews the changes by opening the file
\texttt{html/index.html} in a web browser and verifying the web
pages for the current issue.

The EIC publishes the issue by committing the working copy of the SVN
repository. There will be a delay before the R Journal web site
synchronizes with the SVN repository.

\subsection{Notifying publication}

The EIC verifies that the current issue is online at
\url{journal.r-project.org} and sends the following notifications:
\begin{itemize}
\item The EIC sends an announcement to
  \url{r-annonce@r-project.org}. The list is moderated by Martin
  Maechler so there will be a delay before it is approved.
\item The EIC sends an email to EBSCO at the address 
  \url{Eppubdata@epnet.com} notifying them of the current issue.
  and giving the URL of the R Journal web site.
\item The EIC sends a printed copy of the journal to Thomson Reuters
  (two-sided, colour, bound or stapled in the top-left corner) along
  with the covering letter \texttt{Rjournal-adm/share/Thomson.tex}
  (Thomson are evaluating whether they can pull indexing data directly
  from the web site, which should remove the need to do this).
\end{itemize}

\section{Renewal of the Board}

The search for a new editorial board member should begin in September,
giving enought time to find a replacement before the outgoing member
leaves at the end of December.  Nominations for the new member are
discussed by the editorial board and, if necessary, the advisory
committee.

Once a new member is found, the following steps take place:
\begin{itemize}
\item EIC writes to Martin Maechler requesting SVN access for the new member.
\item The incoming board member requests access to the mailing lists by
visiting \url{https://stat.ethz.ch/mailman/listinfo/r-journal-editors}
and \url{https://stat.ethz.ch/mailman/listinfo/r-journal-advisory}.
\item The outgoing member should remain subscribed to \verb+r-journal-advisory+
\item The EIC may take on outstanding handling editor duties of the
  outgoing member
\end{itemize}

\end{document}

